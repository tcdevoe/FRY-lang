\documentclass{article}
\usepackage{listings}
\usepackage{color}
\usepackage{hyperref}

% Default fixed font does not support bold face
\DeclareFixedFont{\ttb}{T1}{txtt}{bx}{n}{10} % for bold
\DeclareFixedFont{\ttm}{T1}{txtt}{m}{n}{10}  % for normal

\definecolor{deepblue}{rgb}{0,0,0.5}
\definecolor{deepred}{rgb}{0.6,0,0}
\definecolor{deepgreen}{rgb}{0,0.5,0}
\definecolor{codebg}{gray}{0.9}
\definecolor{comment}{gray}{0.4}

\lstset{
backgroundcolor=\color{codebg},
language=Python,
basicstyle=\ttm,
commentstyle=\color{comment}\ttm,
otherkeywords={Join, Sort, delim, Read, Write, ret, Layout, List, Table, true, false },             % Add keywords here
keywordstyle=\ttb\color{deepblue},
stringstyle=\color{deepgreen},
frame=tb,                         % Any extra options here
showstringspaces=false,            
breaklines=true
}

\title{FRY Language Reference}
\author{Tom DeVoe \\ tcd2123@columbia.edu}
\date{\today}

\begin{document}
\maketitle

\tableofcontents

\section{Introduction}
This document serves as a reference manual for the \textbf{FRY} Programming Language. \textbf{FRY} is a language designed for processing delimited text files.


\section{Lexical Conventions}
% In C sections are Tokens, Comments, Identifiers, Keywords, Constants

\subsection{Comments}
Single line comments are denoted by the character, \texttt{\#}. Multi-line comments are opened with \texttt{\#/} and closed with \texttt{/\#}. 
\\
\\
\texttt{\# This is a single line comment}
\\
\\
\texttt{\#/ This is a 
\\
				multi-line comment /\#}
\subsection{Identifiers}
An identifier is a string of letters, digits, and underscores. A valid identifier begins with an letter or an underscore. Identifiers are case-sensitive and can be at most 31 characters long.

\subsection{Keywords}

The following identifiers are reserved and cannot be used otherwise:

\vspace{5 mm}
\texttt{%
\begin{tabular}{ l l l l l }
int & str & float & bool  & Layout \\
List & Table & if & else & elif \\
in & Sort \\
\end{tabular}
}

\subsection{Constants}
There is a constant corresponding to each Primitive data type mentioned in \ref{sec:prims}.

\begin{itemize}
\item \textbf{Integer Constants} - Integer constants are whole base-10 numbers represented by a series of numerical digits (0 - 9). 
\begin{lstlisting}
# Integer Constant Examples
int x = 312342
int y = 111111112
int z = 8
\end{lstlisting}

\item \textbf{Float Constants} - Float constants are similar to Integer constants in that they are base-10 numbers represented by a series of numerical digits. However, floats can also include a decimal separator.
\begin{lstlisting}
# Float Constant Examples
float f1 = 1.158472
float f2 = 2457.89
float f3 = 19999.999999
\end{lstlisting}

\item \textbf{String Constants} - String constants are represented by a series of ASCII characters surrounded by quotation-marks. Certain characters can be escaped inside of Strings with a backslash \textbf{'\'}. These characters are:

\begin{tabular}{ l | l | l }
\textbf{Character} & \textbf{Meaning} \\
\texttt{\textbackslash n } & Newline \\
\texttt{\textbackslash t} & Tab \\
\texttt{\textbackslash  \textbackslash} & Backslash \\
\texttt{\textbackslash " } & Double Quotes \\
\end{tabular}
\\
\begin{lstlisting}
# String Constant Examples
str s1 = "This is \t a string\n"
str s2 = "This. is. also-a-\"string!\""
str s3 = "42"
\end{lstlisting}

\item \textbf{Boolean Constants} - Boolean constants can either have the case-sensitive value \emph{true} or \emph{false}.
\begin{lstlisting}
# Boolean Constant Examples
bool b1 = true
bool b2 = false
\end{lstlisting}

\end{itemize}

\section{Types}
\subsection{Primitive Types}
\label{sec:prims}
\begin{itemize}
% Add in some examples after each 
\item \texttt{int} - 64-bit signed integer value

\item \texttt{str} - An ASCII text value

\item \texttt{float} - A double precision floating-point number

\item \texttt{bool} - A boolean value. Can be either \texttt{true} or \texttt{false}

\end{itemize}

\subsection{Compound Types}

\begin{itemize} 

\item \texttt{List} - an ordered collection of elements of the same data type. Every column in a \emph{Table} is represented as a List

\item \texttt{Layout} - a collection of named data types. Layouts behave similar to structs from C. Once a Layout is constructed, that layout may be used as a data type.  An instance of a Layout is referred to as a \emph{Record} and every table is made up of records of the Layout which corresponds to that table.

\item \texttt{Table} - a representation of a relational table. Every column in a table can be treated as a \emph{List} and every row is a record of a certain \emph{Layout}. Tables are the meat and potatoes of \textbf{FRY} and will be at the center of most programs.

\end{itemize}

\section{Meaning of Identifiers}

\section{Conversions}
% Describes conversion between types

\section{Expressions}
% Describes precedence of expression operators 
% Different types of expressions (Primary expression - a + b ; Postfix Expression - a++, etc.)
% Function Calls, Structure Referenxces (struct, union)
% Multiplicative/Additive Operators, conditional or, comma operator, etc. (other operators)

\section{Declarations}
% Type specifiers, struct union declarations, etc.


\section{Statements}
% Expression Statements, Flow Control, Iteration Stmt (for, while)

\section{Scope}



\end{document}